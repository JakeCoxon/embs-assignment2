\documentclass{article}
\usepackage{hyperref}
\usepackage{amsmath}
\usepackage{mathtools}
\usepackage{float}
\usepackage{xcolor}
\usepackage[a4paper]{geometry}
\usepackage[textsize=small,textwidth=100]{todonotes}


\newcommand{\code}[1]{\texttt{#1}}

\begin{document}
\title{Assessment Two}
\author{Jake Coxon}
\maketitle

\section{}
\subsection{}

\begin{table}[H]
  \centering
  \begin{tabular}{|l|rrrrrrrrrr|}
    \hline
    ~   & A  & B  & C  & D  & E  & F  & G  & H  & I  & J  \\ \hline
    c1  &  3 & -2 &  0 &  0 &  0 &  0 &  0 &  0 &  0 &  0 \\ 
    c2  &  4 &  0 & -1 &  0 &  0 &  0 &  0 &  0 &  0 &  0 \\ 
    c3  &  0 &  4 &  0 & -3 &  0 &  0 &  0 &  0 &  0 &  0 \\ 
    c4  &  0 &  0 &  1 & -2 &  0 &  0 &  0 &  0 &  0 &  0 \\ 
    c5  & -2 &  0 &  0 &  1 &  0 &  0 &  0 &  0 &  0 &  0 \\ 
    c6  &  0 &  0 &  1 &  0 & -1 &  0 &  0 &  0 &  0 &  0 \\ 
    c7  &  2 &  0 &  0 &  0 &  0 & -4 &  0 &  0 &  0 &  0 \\ 
    c8  &  0 &  0 &  0 &  0 &  1 &  0 & -2 &  0 &  0 &  0 \\ 
    c9  &  0 &  0 &  0 &  0 &  0 &  8 & -2 &  0 &  0 &  0 \\ 
    c10 &  0 &  0 &  0 &  6 &  0 &  0 &  0 & -2 &  0 &  0 \\ 
    c11 &  0 &  0 &  0 &  0 &  0 &  0 &  0 &  1 & -3 &  0 \\ 
    c12 &  0 &  0 &  0 &  0 &  0 &  0 &  1 &  0 &  0 & -1 \\ 
    c13 &  0 &  0 &  0 &  0 &  0 &  0 &  0 &  0 &  1 & -1 \\ 
    c14 &  0 &  0 &  0 &  0 &  2 &  0 &  0 &  0 & -X &  0\\ \hline
\end{tabular}
\end{table}

In order the find the consumption rate X of actor I over channel c14 we need to determine the PASS (Periodic Admissible Sequential Schedule).
\[
\bordermatrix{~ & A & B & \cdots & J \cr
              c1 & 3 & -2 & \cdots & 0 \cr
              c2 & 4 & 0 & \cdots & 0 \cr
              \vdots & \vdots & \vdots & \ddots & \vdots \cr
              c14 & 0 & 0 & \cdots & 0 \cr}
    \left( \begin{array}{ccc}
     a \\
     b \\
     \vdots \\
     j
    \end{array} \right) = 0
\]
Which gives
\[3a = 2b, 4a = c, 4b = 3d, c = 2d, d = 2a, c = e, 2a = 4f, e = 2g, 8f = 2g,\]
\[6d = 2g, h = 3i, g = j, i = j, 2e = Xi\]
And therefore
\[12a = 8b = 3c = 6d = 3e = 24f = 6g = 2h = 6i = 6j\]
And $X = 4$
We can get values for $a..j$ by finding the least common integer denominator.
\[a = 2, b = 3, c = 8, d = 4, e = 8, f = 1, g = 4, h = 12, i = 4, j = 4\]

\subsection{}

The sequence can be found by summing the number of tokens displaced for each edge.
\begin{table}[H]
  \centering \tiny

  \begin{tabular}{|l|lllllllllllllllllllllllll|}
    \hline
~  & ~ & A & C & A & B & C & F & E & D & B & C & C & E & H & H & D & H & G & H & H & E & E & G & I & H \\ \hline
c1  & 0 & 3 & 3 & 6 & 4 & 4 & 4 & 4 & 4 & 2 & 2 & 2 & 2 & 2 & 2 & 2 & 2 & 2 & 2 & 2 & 2 & 2 & 2 & 2 & 2 \\
c2  & 0 & 4 & 3 & 7 & 7 & 6 & 6 & 6 & 6 & 6 & 5 & 4 & 4 & 4 & 4 & 4 & 4 & 4 & 4 & 4 & 4 & 4 & 4 & 4 & 4 \\
c3  & 0 & 0 & 0 & 0 & 4 & 4 & 4 & 4 & 1 & 5 & 5 & 5 & 5 & 5 & 5 & 2 & 2 & 2 & 2 & 2 & 2 & 2 & 2 & 2 & 2 \\
c4  & 0 & 0 & 1 & 1 & 1 & 2 & 2 & 2 & 0 & 0 & 1 & 2 & 2 & 2 & 2 & 0 & 0 & 0 & 0 & 0 & 0 & 0 & 0 & 0 & 0 \\
c5  & 4 & 2 & 2 & 0 & 0 & 0 & 0 & 0 & 1 & 1 & 1 & 1 & 1 & 1 & 1 & 2 & 2 & 2 & 2 & 2 & 2 & 2 & 2 & 2 & 2 \\
c6  & 0 & 0 & 1 & 1 & 1 & 2 & 2 & 1 & 1 & 1 & 2 & 3 & 2 & 2 & 2 & 2 & 2 & 2 & 2 & 2 & 1 & 0 & 0 & 0 & 0 \\
c7  & 0 & 2 & 2 & 4 & 4 & 4 & 0 & 0 & 0 & 0 & 0 & 0 & 0 & 0 & 0 & 0 & 0 & 0 & 0 & 0 & 0 & 0 & 0 & 0 & 0 \\
c8  & 0 & 0 & 0 & 0 & 0 & 0 & 0 & 1 & 1 & 1 & 1 & 1 & 2 & 2 & 2 & 2 & 2 & 0 & 0 & 0 & 1 & 2 & 0 & 0 & 0 \\
c9  & 0 & 0 & 0 & 0 & 0 & 0 & 8 & 8 & 8 & 8 & 8 & 8 & 8 & 8 & 8 & 8 & 8 & 6 & 6 & 6 & 6 & 6 & 4 & 4 & 4 \\
c10 & 0 & 0 & 0 & 0 & 0 & 0 & 0 & 0 & 6 & 6 & 6 & 6 & 6 & 4 & 2 & 8 & 6 & 6 & 4 & 2 & 2 & 2 & 2 & 2 & 0 \\
c11 & 0 & 0 & 0 & 0 & 0 & 0 & 0 & 0 & 0 & 0 & 0 & 0 & 0 & 1 & 2 & 2 & 3 & 3 & 4 & 5 & 5 & 5 & 5 & 2 & 3 \\
c12 & 0 & 0 & 0 & 0 & 0 & 0 & 0 & 0 & 0 & 0 & 0 & 0 & 0 & 0 & 0 & 0 & 0 & 1 & 1 & 1 & 1 & 1 & 2 & 2 & 2 \\
c13 & 0 & 0 & 0 & 0 & 0 & 0 & 0 & 0 & 0 & 0 & 0 & 0 & 0 & 0 & 0 & 0 & 0 & 0 & 0 & 0 & 0 & 0 & 0 & 1 & 1 \\
c14 & 0 & 0 & 0 & 0 & 0 & 0 & 0 & 2 & 2 & 2 & 2 & 2 & 4 & 4 & 4 & 4 & 4 & 4 & 4 & 4 & 6 & 8 & 8 & 4 & 4 \\ \hline 
\end{tabular}
  \begin{tabular}{|l|llllllllllllllllllllllllll|}
    \hline
~   & B & C & E & I & C & D & J & J & H & H & H & E & C & I & C & E & D & E & G & H & H & H & I & J & G & J \\ \hline
c1  & 0 & 0 & 0 & 0 & 0 & 0 & 0 & 0 & 0 & 0 & 0 & 0 & 0 & 0 & 0 & 0 & 0 & 0 & 0 & 0 & 0 & 0 & 0 & 0 & 0 & 0 \\
c2  & 4 & 3 & 3 & 3 & 2 & 2 & 2 & 2 & 2 & 2 & 2 & 2 & 1 & 1 & 0 & 0 & 0 & 0 & 0 & 0 & 0 & 0 & 0 & 0 & 0 & 0 \\
c3  & 6 & 6 & 6 & 6 & 6 & 3 & 3 & 3 & 3 & 3 & 3 & 3 & 3 & 3 & 3 & 3 & 0 & 0 & 0 & 0 & 0 & 0 & 0 & 0 & 0 & 0 \\
c4  & 0 & 1 & 1 & 1 & 2 & 0 & 0 & 0 & 0 & 0 & 0 & 0 & 1 & 1 & 2 & 2 & 0 & 0 & 0 & 0 & 0 & 0 & 0 & 0 & 0 & 0 \\
c5  & 2 & 2 & 2 & 2 & 2 & 3 & 3 & 3 & 3 & 3 & 3 & 3 & 3 & 3 & 3 & 3 & 4 & 4 & 4 & 4 & 4 & 4 & 4 & 4 & 4 & 4 \\
c6  & 0 & 1 & 0 & 0 & 1 & 1 & 1 & 1 & 1 & 1 & 1 & 0 & 1 & 1 & 2 & 1 & 1 & 0 & 0 & 0 & 0 & 0 & 0 & 0 & 0 & 0 \\
c7  & 0 & 0 & 0 & 0 & 0 & 0 & 0 & 0 & 0 & 0 & 0 & 0 & 0 & 0 & 0 & 0 & 0 & 0 & 0 & 0 & 0 & 0 & 0 & 0 & 0 & 0 \\
c8  & 0 & 0 & 1 & 1 & 1 & 1 & 1 & 1 & 1 & 1 & 1 & 2 & 2 & 2 & 2 & 3 & 3 & 4 & 2 & 2 & 2 & 2 & 2 & 2 & 0 & 0 \\
c9  & 4 & 4 & 4 & 4 & 4 & 4 & 4 & 4 & 4 & 4 & 4 & 4 & 4 & 4 & 4 & 4 & 4 & 4 & 2 & 2 & 2 & 2 & 2 & 2 & 0 & 0 \\
c10 & 0 & 0 & 0 & 0 & 0 & 6 & 6 & 6 & 4 & 2 & 0 & 0 & 0 & 0 & 0 & 0 & 6 & 6 & 6 & 4 & 2 & 0 & 0 & 0 & 0 & 0 \\
c11 & 3 & 3 & 3 & 0 & 0 & 0 & 0 & 0 & 1 & 2 & 3 & 3 & 3 & 0 & 0 & 0 & 0 & 0 & 0 & 1 & 2 & 3 & 0 & 0 & 0 & 0 \\
c12 & 2 & 2 & 2 & 2 & 2 & 2 & 1 & 0 & 0 & 0 & 0 & 0 & 0 & 0 & 0 & 0 & 0 & 0 & 1 & 1 & 1 & 1 & 1 & 0 & 1 & 0 \\
c13 & 1 & 1 & 1 & 2 & 2 & 2 & 1 & 0 & 0 & 0 & 0 & 0 & 0 & 1 & 1 & 1 & 1 & 1 & 1 & 1 & 1 & 1 & 2 & 1 & 1 & 0 \\
c14 & 4 & 4 & 6 & 2 & 2 & 2 & 2 & 2 & 2 & 2 & 2 & 4 & 4 & 0 & 0 & 2 & 2 & 4 & 4 & 4 & 4 & 4 & 0 & 0 & 0 & 0 \\ \hline 
\end{tabular}
\end{table}



\end{document}
